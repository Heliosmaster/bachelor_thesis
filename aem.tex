\chapter{Alternating Extragradient Method}

\section{Primal-Dual formulation of the image restoration problem}

In section \ref{bayesian_framework}, we observed that image restoration, in the Bayesian framework, is obtained by solving the optimization problem \eqref{map}, with $x \in X$, where $X$ is defined by \eqref{domain}.

Being under the hypotesis of Poisson noise, and focusing as before on the edge-preserving regularization via Total Variation \eqref{defTV}, the so-called \textbf{primal} formulation of the problem is:

\begin{align}
\label{KL-TV}
\min_{x \in X} \left\{ \sum_{i \in S} \left[f_i \log{\frac{f_i}{x_i}}+x_i - f_i\right] + \beta \sum_{i \in S} ||A_i x||_2 \right\}
\end{align}

where $A_i$ is the approximation of the gradient of $x$ at the pixel $i$ as seen in \eqref{matrixA}.

There are a class of methods that require a slightly different formulation of this problem: let us consider, in general, $a \in \R^2$; since the definition of norm in $\ell_2$ can be formulated as:

$$||a||_2 = \sup_{z \in \R^2 : ||z||_2 \leq 1} z^T a$$

we have that

\begin{align}
\label{Aix}
\sum_{i=1}^n ||A_i x||_2 = \sup_{y \in Y} \sum_{i=1}^n y_i^T A_i x \qquad \mbox{ with} \; y_i = \begin{pmatrix} y_1^{(i)} \\
y_2^{(i)}\end{pmatrix}
\end{align}

where $Y = \left\{ y \in \R^{2n} \left |\; ||y_i||_2 \leq 1\;, j=1,...,n\right. \right\}$

Putting

$$ y = \left(y_1^{(1)},y_2^{(1)},y_1^{(2)},y_2^{(2)},\dots,y_1^{(n)},y_2^{(n)}\right)
\qquad A = \begin{pmatrix} A_1 \\
\vdots \\
A_n \end{pmatrix} \qquad x = \begin{pmatrix} x_1 \\
\vdots \\
x_n \end{pmatrix} $$

then \eqref{Aix} can be also written as the matrix product: $\; \displaystyle \sup_{y \in Y} y^T Ax$.

The problem \eqref{KL-TV} can now be rewritten as

$$\min_{x \in X} \left\{ \sum_{i \in S} \left[f_i \log{\frac{f_i}{x_i}}+x_i - f_i\right] + \beta \sup_{y \in Y} y^T Ax \right\}$$

obtaining the discrete \textbf{primal-dual} formulation of the problem \eqref{KL-TV}:

\begin{align}
\label{primal-dual}
\min_{x \in X} \max_{y \in Y} F(x,y)
\end{align}

with

\begin{align}
\label{saddle}
F(x,y) = \left\{ \sum_{i \in S} \left[f_i \log{\frac{f_i}{x_i}}+x_i - f_i\right] + \beta y^T Ax \right\}
\end{align}

The problem  \eqref{KL-TV} has an unique solution $x^* \in X$. Then we can restrict the minimization problem to a closed bounded set $\overline{X} \subset X$, such that $x^* \in \overline{X}$. Since $Y$ is a bounded set and the functional $F$ is convex with respect to $x$ and concave with respect to $y$, the Minimax Theorem in \citep{rockafellar} guarantees the existence of at least one \emph{saddle-point}, that is a point $(x^*,y^*)$ such that

$$F(x^*,y) \leq F(x^*,y^*) \leq F(x,y^*)$$

This class of problems are often found in \emph{Variational Inequality Problems} (VIP).

\section{Extragradient Methods}

As suggested in \citep{aem}, one of the most effective approaches for a solution of saddle-point problems is the class of extragradient-type methods, whose iterative schemes consist in projection steps, along with a suitable choice of a steplength parameter.

A solution $v^* = (x^*,y^*)$ of \eqref{primal-dual} satisfies also the VIP

\begin{align}
\label{vip}
(v-v^*)^T \Phi(v^*) \geq 0 \qquad \forall v \in X \times Y
\end{align}

where

$$\Phi(v) = \Phi(x,y) = \begin{pmatrix}
\nabla_x F(x,y)^T \\
\\
-\nabla_y F(x,y)^T\end{pmatrix}
$$

A solution $v^*$ of \eqref{vip} satisfies the fixed point equation

\begin{align}
\label{proj}
v^* = P_{X \times Y} (v^* - \alpha \Phi(v^*)) \qquad \forall \alpha > 0
\end{align}

where $P$ denotes the orthogonal projection operator, defined as

$$y = P_\Omega (x) = \argmin_{z \in \Omega} ||x-z||_2^2$$

Since the domain $X \times Y$ has a separable structure, \eqref{proj} holds if and only if

\begin{align}
\label{proj2-1} x^* &= P_X (x^* - \alpha \nabla_x F(x^*,y^*)) \\
\notag \\
\label{proj2-2} y^* &= P_Y (y^* - \alpha \nabla_y F(x^*,y^*))
\end{align}

Both these two projections are fairly easy to compute: \eqref{proj2-1} is a simple thresolding, while in \eqref{proj2-2} the projection onto $Y$ is defined as $(P_Y (y))_k = s_ky_k$, with

$$s_{2i-1} = s_{2i} = \dfrac{1}{\max\left\{1,\sqrt{y_{2i-1}^2+y_{2i}^2}\right\}} \qquad i=1,\dots,n$$

Due to the structure of the constraints in \eqref{primal-dual}, the projection methods are attractive; however, the effectiveness of a projection method as such is deeply related to the choice of the steplength parameter.

Since $F$ is concave with respect to $y$ and convex with respect to $x$, $\Phi(v)$ is a monotone function (see Theorem \ref{app:monotone}). A well known scheme to solve monotone variational inequality problems as \eqref{vip} is the extragradient method defined by the iteration:

\begin{align}
\label{extragradient_method}
\begin{aligned}
\bar{v}^{(k)} &= P_{X \times Y} \left(v^{(k)}-\alpha_k \Phi\left(v^{(k)}\right)\right) \\
v^{(k+1)} &= P_{X \times Y} \left(v^{(k)} - \alpha_k \Phi\left(\bar{v}^{(k)}\right)\right)
\end{aligned}
\end{align}

where $\alpha_k$ is the steplength.

There are mainly two strategies about the choice of such steplength $\alpha_k$: it can be held fixed and, usually, its value depends on an \emph{a priori} estimate of the Lipschitz constant of $\Phi$; otherwise it can be adaptively updated at each step, satisfying each time the sufficent condition for convergence of the method. The second choice is often the best, since an \emph{a priori} estimate of Lipschitz constant could be difficult.

An adaptive rule to select $\alpha_k$ is to define the steplength so that the following condition is satisfied:

$$1-\alpha_k^2 \dfrac{\left|\left|\Phi\left(v^{(k)}\right) - \Phi\left(\bar{v}^{(k)}\right)\right|\right|_2^2}{\left|\left|v^{(k)}-\bar{v}^{(k)}\right|\right|_2} > 0 $$

In practice, such $\alpha_k$ is determined by a backtracking procedure which guaratnees the convergence of the method under suitable monotonicity and Lipschitz conditions on $\Phi$ \citep{khobo}.

\section{Alternating extragradient method}

Following the strategy in \citep{khobo}, we can state a method for saddle point problems. The iterate $v^{(k)}$ is split in two successive iterates $x^{(k)}$ and $y^{(k)}$.

The choice of the steplength is adaptive and, furthermore, there is not explicit use of the Lipschitz constant of $\Phi$ , but only a local approximation of it is used.

The method is the following:

\begin{alignat}{2}
&\label{aem-1} \bar{y}^{(k)} &&= P_Y\left(y^{(k)} + \alpha_k \nabla_y F(x^{(k)},y^{(k)})\right) \\
\notag \\
&\label{aem-2} x^{(k+1)} &&= P_X\left(x^{(k)} + \alpha_k \nabla_x F\left(x^{(k)},\bar{y}^{(k)}\right)\right) \\
\notag \\
&\label{aem-3} y^{(k+1)} &&= P_Y\left(y^{(k)} + \alpha_k \nabla_y F\left(x^{(k+1)},y^{(k)}\right)\right)
\end{alignat}

with the steplength $\alpha_k$ is chosen in the bounded interval $[\alpha_{min}, \alpha_{max}]$ with $0<\alpha_{min}<\alpha_{max}$ and

\begin{align}
\left\{ \begin{array}{lll}
1-2\alpha_k A_k - 2 \alpha_k^2 B_k^2 & \geq & \epsilon \\
\\
1-2\alpha_k C_k & \geq & \epsilon \end{array} \right.
\end{align}


where $\epsilon \in (0,1)$ is costant and

\begin{align}
\begin{aligned}
A_k &= \dfrac{\left|\left|\nabla_x F\left(x^{(k+1)},\bar{y}^{(k)}\right) - \nabla_x F\left(x^{(k)},\bar{y}^{(k)}\right)\right|\right|_2}{\left|\left|x^{(k+1)}-x^{(k)}\right|\right|_2} \\
\\
B_k &= \dfrac{\left|\left|\nabla_y F\left(x^{(k+1)},y^{(k)}\right) - \nabla_y F\left(x^{(k)},y^{(k)}\right)\right|\right|_2}{\left|\left|x^{(k+1)}-x^{(k)}\right|\right|_2} \\
\\
C_k &= \dfrac{\left|\left|\nabla_y F\left(x^{(k+1)},y^{(k)}\right) - \nabla_y F\left(x^{(k+1)},\bar{y}^{(k)}\right)\right|\right|_2}{\left|\left|y^{(k)}-\bar{y}^{(k)}\right|\right|_2}
\end{aligned}
\end{align}

This scheme substantially consist in successive gradient ascent \eqref{aem-1} and descent \eqref{aem-2} steps, followed by an extragradient step \eqref{aem-3}. The method can also be written exchanging the roles of $x$ and $y$.

In \citep{aem} it is outlined that such scheme has two main advantages to other methods:

\begin{itemize}
 \item it is defined only by means of explicits steps;
 \item the convergence of the scheme is guaranteed without strict convexity assumptions, thanks to the extragradient step \eqref{aem-3}.
\end{itemize}

In particular, it is shown that the convergence is guaranteed by an appropriate selection of $\alpha_k$, and it is shown a self-adjusting backtracking procedure to compute it.

\section{Algorithm AEM}

This is the scheme of the algoritm \eqref{aem-1}-\eqref{aem-3}:

\begin{algorithm}[H]
\caption{Alternating Extragradient Method (AEM)}
\begin{algorithmic}
\STATE Choose $\left(x^{(0)},y^{(0)}\right) \in X \times Y$
\STATE Set $\theta$
\STATE Set $\epsilon \in (0,1)$
\STATE Set $\alpha_{max} > 0$
\STATE
\end{algorithmic}

Step 1:
\begin{algorithmic}
\STATE Choose $\alpha \leq \alpha_{max}$
\STATE
\end{algorithmic}

Step 2: Compute tentative points
\begin{algorithmic}
\STATE $ \bar{y}^+ \gets P_Y \left(y^{(k)} + \alpha \nabla_y F\left(x^{(k)},y^{(k)}\right)\right)$
\STATE
\STATE $ x^+ \gets P_X \left(x^{(k)} - \alpha \nabla_x F\left(x^{(k)},\bar{y}^+\right)\right)$
\STATE
\STATE $A \gets \dfrac{||\nabla_x F\left(x^+,\bar{y}^+\right) - \nabla_x F\left(x^{(k)},\bar{y}^+\right)||_2}{||x^+-x^{(k)}||_2} $
\STATE
\STATE $B \gets \dfrac{||\nabla_y F\left(x^+,y^{(k)}\right) - \nabla_y F\left(x^{(k)},y^{(k)}\right)||_2}{||x^+-x^{(k)}||_2} $
\STATE
\STATE $C \gets \dfrac{||\nabla_y F\left(x^+,y^{(k)}\right) - \nabla_y F\left(x^+,\bar{y}^+\right)||_2}{||y^{(k)}-\bar{y}^+||_2}$
\STATE
\STATE $\bar{\alpha} \gets \left\{ \begin{array}{ll}
\min \left\{\dfrac{\sqrt{A^2+2B^2(1-\epsilon)}-A}{2B^2}, \dfrac{1-\epsilon}{2C}\right\} & \mbox{ if }\; B>0, C>0 \\
\\
\min \left\{\dfrac{1-\epsilon}{2A},\dfrac{1-\epsilon}{2C}\right\} & \mbox{ if }\; A > 0 , C>0, B=0 \\
\\
\dfrac{\sqrt{A^2+2B^2(1-\epsilon)}-A}{2B^2} & \mbox{ if }\; B>0,C=0\\
\\
\dfrac{1-\epsilon}{2C} & \mbox{ if }\;A=0,C>0,B=0\\
\\
\dfrac{1-\epsilon}{2A} & \mbox{ if }\; A>0,C=0,B=0\\
\\
\alpha & \mbox{ otherwise}
\end{array} \right. $
\STATE
\end{algorithmic}

Step 3: Check convergence condition
\begin{algorithmic}
\IF{$\alpha \leq \bar{\alpha}$}
\STATE $\alpha_k = \alpha$
\STATE $\bar{y}^{(k)} = \bar{y}^+$
\STATE $x^{(k+1)} = x^+$
\ELSE
\STATE $\alpha \gets \min\{\bar{\alpha},\theta \alpha\}$
\STATE go to Step 2
\ENDIF
\end{algorithmic}

Step 4:
\begin{algorithmic}
\STATE $y^{(k+1)} = P_Y \left(y^{(k)} + \alpha_k \nabla_y F\left(x^{(k+1)},y^{(k)}\right)\right)$
\end{algorithmic}
\end{algorithm}


In particular, for $F(x,y)$ in \eqref{saddle}, we have that

\begin{align*}
A_k &= \dfrac{\left|\left|\dfrac{f}{x^{(k)}} - \dfrac{f}{x^{(k+1)}}\right|\right|_2}{||x^{(k+1)}-x^{(k)}||_2} \\
\\
B_k &= \dfrac{\left|\left|\beta Ax^{(k+1)} - \beta A x^{(k)}\right|\right|_2}{\left|\left|x^{(k+1)}-x^{(k)}\right|\right|_2} \\
\\
C_k &= 0
\end{align*}


